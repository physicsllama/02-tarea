\documentclass[letter, 11pt]{article}
%% ================================
%% Packages =======================
\usepackage[utf8]{inputenc}      %%
\usepackage[T1]{fontenc}         %%
\usepackage{lmodern}             %%
\usepackage[spanish]{babel}      %%
\usepackage{fullpage}            %%
\usepackage{fancyhdr}            %%
\usepackage{graphicx}            %%
\usepackage{amsmath}             %%
\usepackage{color}               %%
\usepackage{mdframed}            %%
\usepackage[colorlinks]{hyperref}%%
%% ================================
%% ================================

%% ================================
%% Page size/borders config =======
\setlength{\oddsidemargin}{0in}  %%
\setlength{\evensidemargin}{0in} %%
\setlength{\marginparwidth}{0in} %%
\setlength{\marginparsep}{0in}   %%
\setlength{\voffset}{-0.5in}     %%
\setlength{\hoffset}{0in}        %%
\setlength{\topmargin}{0in}      %%
\setlength{\headheight}{54pt}    %%
\setlength{\headsep}{1em}        %%
\setlength{\textheight}{8.5in}   %%
\setlength{\footskip}{0.5in}     %%
%% ================================
%% ================================

%% =============================================================
%% Headers setup, environments, colors, etc.
%%
%% Header ------------------------------------------------------
\fancypagestyle{firstpage}
{
  \fancyhf{}
  \lhead{\includegraphics[height=4.5em]{LogoDFI.jpg}}
  \rhead{FI3104-1 \semestre\\
         Métodos Numéricos para la Ciencia e Ingeniería\\
         Prof.: \profesor}
  \fancyfoot[C]{\thepage}
}

\pagestyle{plain}
\fancyhf{}
\fancyfoot[C]{\thepage}
%% -------------------------------------------------------------
%% Environments -------------------------------------------------
\newmdenv[
  linecolor=gray,
  fontcolor=gray,
  linewidth=0.2em,
  topline=false,
  bottomline=false,
  rightline=false,
  skipabove=\topsep
  skipbelow=\topsep,
]{ayuda}
%% -------------------------------------------------------------
%% Colors ------------------------------------------------------
\definecolor{gray}{rgb}{0.5, 0.5, 0.5}
%% -------------------------------------------------------------
%% Aliases ------------------------------------------------------
\newcommand{\scipy}{\texttt{scipy}}
%% -------------------------------------------------------------
%% =============================================================

%% =============================================================================
%% CONFIGURACION DEL DOCUMENTO =================================================
%% Llenar con la información pertinente al curso y la tarea
%%
\newcommand{\tareanro}{2}
\newcommand{\fechaentrega}{4/10/2018 23:59 hrs}
\newcommand{\semestre}{2018B}
\newcommand{\profesor}{Valentino González}
%% =============================================================================
%% =============================================================================


\begin{document}
\thispagestyle{firstpage}

\begin{center}
  {\uppercase{\LARGE \bf Tarea \tareanro}}\\
  Fecha de entrega: \fechaentrega
\end{center}


%% =============================================================================
%% ENUNCIADO ===================================================================
\noindent{\large \bf Problema 1 (50\%)}

Ud. trabaja para una compañia eléctrica cableando torres de tensión. Suponga
que las torres de tensión están separadas por 20 metros y que la compañia tiene
una política estricta: el cable debe caer 7.5 metros en el punto medio entre
dos torres consecutivas. La compañia le ha pedido a Ud. que calcule el largo
que debe tener el cable necesario.

Tras investigar un poco Ud. encuentra que la función que describe la forma del
cable colgado de dos extremos se llama catenaria y tiene la siguiente forma:

$$ f(x) = \frac{a}{2} \left( e^{(x-x_0)/2} + e^ {(x-x_0)/2} \right) $$

\noindent donde $x_0$ es el punto mínimo de la catenaria (el punto medio entre
las dos torres en este caso) y $a$ es un parámetro que controla cuanto se
``abre'' la catenaria.

\begin{ayuda}
  \small
  \noindent {\bf Ayuda.}
  Recuerde que el largo de una curva entre los puntos $x_1$ y $x_2$ está dado
  por:

  $$ l = \int_{x_1}^{x_2}dx\sqrt{1 + f'(x)^2} $$

\end{ayuda}

Para este problema se pide que Ud. implemente su propio algoritmo para
encontrar raíces de una función (lo necesitará para determinar el parámetro $a$
de la catenaria). Justifique su elección de algoritmo. Para la integral puede
utilizar cualquier algoritmo que estime conveniente y no es necesario que
escriba uno Ud. mismo (puede usar uno de \scipy, por ejemplo).


%% FIN ENUNCIADO ===============================================================
%% =============================================================================

\end{document}
