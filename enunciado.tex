\documentclass[letter, 11pt]{article}
%% ================================
%% Packages =======================
\usepackage[utf8]{inputenc}      %%
\usepackage[T1]{fontenc}         %%
\usepackage{lmodern}             %%
\usepackage[spanish]{babel}      %%
\usepackage{fullpage}            %%
\usepackage{fancyhdr}            %%
\usepackage{graphicx}            %%
\usepackage{amsmath}             %%
\usepackage{color}               %%
\usepackage{mdframed}            %%
\usepackage[colorlinks]{hyperref}%%
%% ================================
%% ================================

%% ================================
%% Page size/borders config =======
\setlength{\oddsidemargin}{0in}  %%
\setlength{\evensidemargin}{0in} %%
\setlength{\marginparwidth}{0in} %%
\setlength{\marginparsep}{0in}   %%
\setlength{\voffset}{-0.5in}     %%
\setlength{\hoffset}{0in}        %%
\setlength{\topmargin}{0in}      %%
\setlength{\headheight}{54pt}    %%
\setlength{\headsep}{1em}        %%
\setlength{\textheight}{8.5in}   %%
\setlength{\footskip}{0.5in}     %%
%% ================================
%% ================================

%% =============================================================
%% Headers setup, environments, colors, etc.
%%
%% Header ------------------------------------------------------
\fancypagestyle{firstpage}
{
  \fancyhf{}
  \lhead{\includegraphics[height=4.5em]{LogoDFI.jpg}}
  \rhead{FI3104-1 \semestre\\
         Métodos Numéricos para la Ciencia e Ingeniería\\
         Prof.: \profesor}
  \fancyfoot[C]{\thepage}
}

\pagestyle{plain}
\fancyhf{}
\fancyfoot[C]{\thepage}
%% -------------------------------------------------------------
%% Environments -------------------------------------------------
\newmdenv[
  linecolor=gray,
  fontcolor=gray,
  linewidth=0.2em,
  topline=false,
  bottomline=false,
  rightline=false,
  skipabove=\topsep
  skipbelow=\topsep,
]{ayuda}
%% -------------------------------------------------------------
%% Colors ------------------------------------------------------
\definecolor{gray}{rgb}{0.5, 0.5, 0.5}
%% -------------------------------------------------------------
%% Aliases ------------------------------------------------------
\newcommand{\scipy}{\texttt{scipy}}
%% -------------------------------------------------------------
%% =============================================================

%% =============================================================================
%% CONFIGURACION DEL DOCUMENTO =================================================
%% Llenar con la información pertinente al curso y la tarea
%%
\newcommand{\tareanro}{2}
\newcommand{\fechaentrega}{4/10/2018 23:59 hrs}
\newcommand{\semestre}{2018B}
\newcommand{\profesor}{Valentino González}
%% =============================================================================
%% =============================================================================


\begin{document}
\thispagestyle{firstpage}

\begin{center}
  {\uppercase{\LARGE \bf Tarea \tareanro}}\\
  Fecha de entrega: \fechaentrega
\end{center}


%% =============================================================================
%% ENUNCIADO ===================================================================
\noindent{\large \bf Problema 1 (50\%)}

Ud. trabaja para una compañia eléctrica cableando torres de tensión. Suponga
que las torres de tensión están separadas por 20 metros y que la compañia tiene
una política estricta: el cable debe caer 7.5 metros en el punto medio entre
dos torres consecutivas. La compañia le ha pedido a Ud. que calcule el largo
que debe tener el cable necesario.

Tras investigar un poco Ud. encuentra que la función que describe la forma del
cable colgado de dos extremos se llama catenaria y tiene la siguiente forma:

$$ f(x) = \frac{a}{2} \left( e^{(x-x_0)/2} + e^ {(x-x_0)/2} \right) $$

\noindent donde $x_0$ es el punto mínimo de la catenaria (el punto medio entre
las dos torres en este caso) y $a$ es un parámetro que controla cuanto se
``abre'' la catenaria.

\begin{ayuda}
  \small
  \noindent {\bf Ayuda.}
  Recuerde que el largo de una curva entre los puntos $x_1$ y $x_2$ está dado
  por:

  $$ l = \int_{x_1}^{x_2}dx\sqrt{1 + f'(x)^2} $$

\end{ayuda}

Para este problema se pide que Ud. implemente su propio algoritmo para
encontrar raíces de una función (lo necesitará para determinar el parámetro $a$
de la catenaria). Justifique su elección de algoritmo. Para la integral puede
utilizar cualquier algoritmo que estime conveniente y no es necesario que
escriba uno Ud. mismo (puede usar uno de \scipy, por ejemplo).


\vspace{1.5em}
\noindent{\large \bf Problema 2 (50\%)}

Considere las siguientes funciones:
% \begin{equation}
  \begin{align}
    F_1(x, y) &= x^4 + y^4 - 15 \label{eq1}\\
    F_2(x, y) &= x^3y - xy^3 - y/2 - 1.RRR \label{eq2}
  \end{align}
% \end{equation}

\noindent donde $RRR$ representa los últimos 3 dígitos de su RUT (antes del
dígito verificador). Encuentre todos los puntos $(x, y)$ para los cuales las
funciones se hacen simultáneamente cero.

\begin{ayuda}
  \small
  \noindent {\bf Ayuda.}

  Se recomienda partir graficando las funciones. Para visualizar estas
  funciones se requiere alguna forma de gráfico 3D. Una forma efectiva de
  visualización para este caso es un plot de superficie en que la tercera
  dimensión está representada por un mapa de colores combinada con un mapa de
  contornos (ver ejemplo adjunto \texttt{contourplot.py}). En particular, en
  este caso estamos interesados en el contorno al nivel cero pues este
  representa todos los puntos en que una determinada función se hace cero .

  Al hacer lo anterior se dará cuenta que el nivel cero de $F_1$ es una curva
  cerrada. Una idea que le puede servir, es intentar recorrer esa curva cerrada
  buscando los puntos en que la otra función, $F_2$, se hace cero. Es decir,
  Ud.  puede implementar un algoritmo similar al visto en clase para la
  bisección, pero que en vez de acercar los puntos $a$ y $b$ en 1D, lo haga a
  lo largo de la curva determinada por $F_1(x, y)=0$. Para eso necesitará
  encontrar una forma paramétrica de dicha curva (hint: $x(t) \propto
  sign(sin(t)) * \sqrt[4]{sin^2(t)}$; piense en $y(t)$ y el dominio de t).

\end{ayuda}

Para esta pregunta no es necesario que implemente su propio algoritmo para la
búsquea de raíces, puede usar alguno de \scipy~u otro que Ud. estime
conveniente (o programar el suyo propio si lo desea). Explique su elección.

Es importante que describa la estrategia utilizada para resolver el problema.
En particular, explique cómo eligió los valores iniciales de $a$ y $b$ para el
método de la bisección para cada uno de los ceros.

\vspace{1em}
\noindent{\bf Instrucciones importantes.}
\begin{itemize}

  \item Utilice \texttt{git} durante el desarrollo de la tarea para mantener un
      historial de los cambios realizados. La siguiente
      \href{https://education.github.com/git-cheat-sheet-education.pdf}{cheat
      sheet} le puede ser útil. Esta vez requeriremos que haya al menos 2
      commits suyos a lo largo del desarrollo pero no revisaremos que tengan
      mucho sentido. Lo importante, por ahora, es que aprenda la mecánica.

  \item La tarea se entrega subiendo su trabajo a github. Clone este
      repositorio (el que está en su propia cuenta privada), trabaje en el
      código y en el informe (idealmente haga \texttt{commits} a medida que
      progresa) y cuando haya terminado asegúrese de hacer un último
      \texttt{commit} y luego un \texttt{push} para subir todo su trabajo a
      github.

  \item El informe debe ser entregado en formato \texttt{pdf}, este debe ser
      claro sin información de más ni de menos. Esto es importante, no escriba
      de más, esto no mejorará su nota sino que al contrario. Asegúrese de
      utilizar figuras efectivas y tablas para resumir sus resultados. 4 a 5
      páginas son más que suficientes para este informe (incluyendo figuras y
      tablas si corresponde).
  \item {\bf Revise su ortografía.}

  \item No olvide indicar su RUT en el informe.

\end{itemize}

%% FIN ENUNCIADO ===============================================================
%% =============================================================================

\end{document}
